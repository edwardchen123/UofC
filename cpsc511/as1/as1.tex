\documentclass{article}
\usepackage{parskip}
\usepackage{amsmath}
\usepackage{amssymb}
\usepackage{cancel}
\usepackage{algorithmic}
\usepackage{xfrac}

\setlength{\parindent}{0cm}

\begin{document}

\title{CPSC 511 \\ Assignment 1}
\author{Andrew Helwer}
\date{October 2012}
\maketitle

\section{Question 1 - Closure Under Union}

\textbf{1)}

Let $M_1$ be a TM such that $M_1(x) = 1 \Leftrightarrow x \in L_1$ which runs in $DTIME(t(|x|))$, and let $M_2$ be a TM such that $M_2(y) = 1 \Leftrightarrow y \in L_2$ which runs in $DTIME(t(|y|))$. 
$L_3 = L_1 \cup L_2$ is the set of input strings which are in $L_1$ or are in $L_2$.
Consider a new TM, $M_3$.
The first stage of $M_3$ is identical to $M_1$.
So, on input $z$, $M_3$ calculates $r_1 = M_1(z)$, in $DTIME(t(|z|))$.
Next, $M_3$ runs $z$ on its second stage, which is identical to $M_2$.
So, $M_3$ calculates $r_2 = M_2(z)$ in $DTIME(t(|z|))$.
If $r_1 \vee r_2 = 1$, $M_3$ prints 1 to the output tape and halts.
If $r_1 \vee r_2 = 0$, $M_3$ prints 0 to the output tape and halts.
$M_3$ determines membership in $L_1 \cup L_2$ in $2 \cdot t(n)$ time, so in $DTIME(t(n))$.
Thus if $L_1, L_2 \in DTIME(t(n))$ then $L_1 \cup L_2 \in DTIME(t(n))$.

\textbf{2)}

Consider as a counterexample languages $L_1 = \{0, 1\}$ and $L_2 = \{x \in \{0\}^*\}$. 
Then $L_1 \in DTIME(c), c \in \mathbb{N}$ and $L_2 \in DTIME(|X|)$ for input $x$. 
The TM accepting $L_1 \cup L_2$ would have to read $x$ to determine its inclusion in $L_2$, which takes at least $|x|$ steps. 
Thus $L_1 \cup L_2 \not \in DTIME(min(t_1, t_2))$ since $min(t_1, t_2)$ is $c$.

\section{Question 2 - OMPL and Logarithmic Space}

The only rule which cannot be described using only states and transitions (with no additional workspace used) is $O$.
The rule $O$ requires the use of two counters, $c_1$ and $c_2$.
$c_1$ increments every time the read head on the input finds an $<outline>$ character.
$c_2$ increments every time the read head on the input finds an $<outline/>$ character.
If an $</outline>$ character is read, the input is immediately rejected unless $c_1 > 0$ and $c_2 > 0$.
$c_1$ and $c_2$ are both then decremented.
Once the end of the string is reach, assuming all other conditions are satisfied, the input will be accepted only if $c_1 = 0$ (with no restrictions on value of $c_2$).
Assuming encoding for $c_1$ and $c_2$ that is not isomorphic to unary, the counters will take $2 \cdot log(n)$ space, where $n$ is the length of the input.
Thus $L_{OPML} \in L$.

\section{Question 3 - Padding}

\section{Oracles and PSPACE}

\end{document}

