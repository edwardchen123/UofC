\documentclass{article}
\usepackage{parskip}
\usepackage{amsmath}
\usepackage{amssymb}
\usepackage{cancel}
\usepackage{algorithmic}
\usepackage{xfrac}

\setlength{\parindent}{0cm}

\begin{document}

\title{CPSC 413 \\ Assignment 3 - Divide and Conquer}
\author{Andrew Helwer}
\date{March 2011}
\maketitle

\section{Bank Fraud Detection}

This problem is simply a restatement of the majority element problem - given a
collection of elements of size $n$, does there exist an element equivalent to
more than $n/2$ elements of the collection? The obvious (and correct) algorithm
is to simply check equality between all elements of the collection, but this
requires $O(n^2)$ checks no matter how many heuristic optimizations are
included.

If a divide and conquer algorithm is employed, however, the algorithm becomes
much faster - on the order of $O(n \log n)$, as we shall prove. An even faster
algorithm exists using the Monte Carlo method, running in $O(n)$. 

The name of the function implementing the algorithm will be $majElem(S)$, which
returns $S' \subseteq S$ if a majority element exists. If $x \in S'$ then $x$ is
equivalent to the majority element. If no majority element exists, $\emptyset$
is returned. The answer to the original problem is Yes if a nonempty set is
returned, and No otherwise.

If two cards $a$ and $b$ are found equivalent using the card scanner, write $(a
\sim b)$. Note that by the transitivity property of an equivalence relation, if
$(a \sim b)$ and $(b \sim c)$, then $(a \sim c)$.

\textbf{1) Describe how to solve trivial instances:}

The trivial case occurs when $n=1$ - that is, there exists only a single element
in the set. Clearly, that element is the majority element of its set. So
$majElem(\{a\})$ returns $\{a\}$.

\textbf{2) Describe how sub-instances should be formed from nontrivial ones:}

Consider the case for an arbitrary set of elements, $S$, of size $n>1$. A
sub-instance of the problem will be constructed by forming two sets $S_L$ and
$S_R$ that partition $S$, where $|S_L| = \sfrac{n}{2}$ and $|S_R| = n - |S_L|$.
The function will then recurse on $S_L$ and $S_R$ by calling $majElem(S_L)$ and
$majElem(S_R)$.

\textbf{3) Describe how a solution is formed from sub-instance solutions:}

Claim 3.1: If $S$ contains a majority element $a$, then $a$ is a majority
element in $S_L$ or $a$ is a majority element in $S_R$.

Proof: Assume that $S$ contains a majority element $a$, where $|S|=n$. Then
$a$ is equivalent to more than $\sfrac{n}{2}$ elements of $S$. If $S$ is
partitioned into
two disjoint sets $S_L$ and $S_R$, we have:
\begin{equation}
\label{eq:sumSize}
|S_L|+|S_R|=n
\end{equation}
Since $S = S_L \sqcup S_R$, $a$ must be equivalent to more than $\sfrac{n}{2}$
elements
of $S_L \sqcup S_R$, so we have:
\begin{equation}
\label{eq:sumEquiv}
|\{s \in S_L : a \sim s\}| + |\{s \in S_R : a \sim s\}| > \frac{n}{2}
\end{equation}
Without loss of generality, consider the case where $a$ is not a majority
element of $S_L$. Then:
\begin{align*}
|\{s \in S_L : a \sim s\}| &\leq \frac{|S_L|}{2} \\
|\{s \in S_L : a \sim s\}| - \frac{n}{2} &\leq \frac{|S_L|}{2} - \frac{n}{2} \\
\frac{n}{2} - |\{s \in S_L : a \sim s\}| &\geq \frac{n}{2} - \frac{|S_L|}{2} \\
|\{s \in S_R : a \sim s\}| &> \frac{n}{2} - \frac{|S_L|}{2}
&\text{by eqn (\ref{eq:sumEquiv})} \\
|\{s \in S_R : a \sim s\}| &> \frac{n}{2} - \frac{n-|S_R|}{2}
&\text{by eqn (\ref{eq:sumSize})} \\
|\{s \in S_R : a \sim s\}| &> \frac{|S_R|}{2}
\end{align*}
Therefore if $a$ is not a majority element of $S_L$ then $a$ must be a majority
element of $S_R$. By symmetry, if $a$ is not a majority element of $S_R$ then
$a$ must be a majority element of $S_L$. Thus we have proved Claim 3.1.
$\square$

Claim 3.2: If $majElem(S_L)$ returns the nonempty set $S_L' \subseteq S_L$,
$majElem(S_R)$ returns $\emptyset$, and
\begin{equation}
\label{eq:notMaj}
|S_L' \sqcup \{s \in S_R : a \sim s, a \in S_L'\}| \leq \frac{|S|}{2}
\end{equation}
then $S$ does not contain a majority element.

Proof: Assume the antecedent. We wish to prove that $S$ then does not contain a
majority element, so assume that $S$ does contain a majority element, $a$, where
$|S|=n$. Then by Claim 3.1, $a$ is a majority element of $S_L$ or $a$ is a
majority element of $S_R$. Since $S_R$ does not contain a majority element,
$S_L$ must contain a majority element. As it is impossible for there to be two
majority elements in a set, the majority element of $S_L$ must be $a$, where
$a$ is equivalent to every element in $S_L'$. Note:
\begin{align*}
|\{s \in S_L : a \sim s\}| + |\{s \in S_R : a \sim s\}| &> \frac{n}{2}
&\text{by eqn (\ref{eq:sumEquiv})} \\
|S_L'| + |\{s \in S_R : a \sim s\}| &> \frac{n}{2} \\
|S_L' \sqcup \{s \in S_R : a \sim s\}| &> \frac{n}{2}
\end{align*}
However, by equation ($\ref{eq:notMaj}$), this cannot be true, and so by
contradiction $S$ must not contain a majority element. Thus we have proved
Claim 3.2. $\square$

Claim 3.3: If $majElem(S_L)$ returns the nonempty set $S_L' \subseteq S_L$,
$majElem(S_R)$ returns the nonempty set $S_R' \subseteq S_R$, and
\begin{align*}
|S_L' \sqcup \{s \in S_R : a \sim s, a \in S_L'\}| &\leq \frac{|S|}{2} \\
|S_R' \sqcup \{s \in S_L : a \sim s, a \in S_R'\}| &\leq \frac{|S|}{2}
\end{align*}
then $S$ does not contain a majority element.

Proof: Assume the antecedent. We wish to prove that $S$ then does not contain a
majority element, so assume that $S$ does contain a majority element, $a$, where
$|S|=n$. Then by Claim 3.1, $a$ is a majority element of $S_L$ or $a$ is a
majority element of $S_R$. So we have two cases:
\begin{itemize}
\item Case One: $a$ is a majority element of $S_L$. Then $a$ is equivalent
to every element of $S_L'$, and this case is equivalent to the one considered
in Claim 3.2, so a contradiction arises.
\item Case Two: $a$ is a majority element of $S_R$. Then $a$ is equivalent
to every element of $S_R'$, and this case is symmetric to the one considered
in Claim 3.2, so a contradiction arises.
\end{itemize}
Since a contradiction results in every case when $S$ is assumed to contain a
majority element, $S$ must not contain a majority element. Thus we have proved
Claim 3.3. $\square$

Upon recursing on $S_L$ and $S_R$, where $|S|=n$, there are four possible cases:

\begin{enumerate}
\item $majElem(S_L)$ returns $S_L'$ and $majElem(S_R)$ returns $\emptyset$

Execute the following algorithm, for any $b \in S_L'$:
\begin{algorithmic}
\FOR {$a \in S_R$}
	\IF {$(a \sim b)$}
		\STATE $S_L' \gets S_L' \cup \{a\}$
	\ENDIF
\ENDFOR
\end{algorithmic}
If $|S_L'| > \sfrac{n}{2}$, then $S$ contains a majority element by definition,
and since a set may only have one majority element $S_L'$ should be returned.
If $|S_L'| \leq \sfrac{n}{2}$, then by Claim 3.2 no majority element exists in
$S$, so $\emptyset$ should be returned.

\item $majElem(S_L)$ returns $\emptyset$ and $majElem(S_R)$ returns $S_R'$

This case is symmetric to case 1, with Claim 3.2 able to be used as the proof
for it is also symmetric.

\item $majElem(S_L)$ returns $S_L'$ and $majElem(S_R)$ returns $S_R'$

Execute the following algorithm, for any $b \in S_L'$:
\begin{algorithmic}
\FOR {$a \in S_R$}
	\IF {$(a \sim b)$}
		\STATE $S_L' \gets S_L' \cup \{a\}$
	\ENDIF
\ENDFOR
\end{algorithmic}
If $|S_L'| > \sfrac{n}{2}$, then $S$ contains a majority element by definition,
and since a set may only have one majority element $S_L'$ should be returned.
If $|S_L'| \leq \sfrac{n}{2}$, then execute the following algorithm, for any $b
\in S_R'$:
\begin{algorithmic}
\FOR {$a \in S_L$}
	\IF {$(a \sim b)$}
		\STATE $S_R' \gets S_R' \cup \{a\}$
	\ENDIF
\ENDFOR
\end{algorithmic}
If $|S_R'| > \sfrac{n}{2}$, then $S$ contains a majority element by definition,
and since a set may only have one majority element $S_R'$ should be returned.
If $|S_R'| \leq \sfrac{n}{2}$, then by Claim 3.3 no majority element exists in
$S$, so $\emptyset$ should be returned.

\item Both $majElem(S_L)$ and $majElem(S_R)$ return $\emptyset$

By the converse of Claim 3.1, $S$ does not contain a majority element, and so
$\emptyset$ should be returned.

\end{enumerate}

\textbf{4) Describe a recursive algorithm implementation:}

$majElem(S)$
\begin{algorithmic}[1]
\IF {$|S|=1$} 
	\RETURN $S$
\ELSE 
	\STATE Split $S$ into $S_L$ and $S_R$
	\STATE $S_L' \gets majElem(S_L)$ and $S_R' \gets majElem(S_R)$
	\IF {$S_L' \neq \emptyset$ and $S_R' = \emptyset$}
		\STATE $S_L' \gets appendEq(S_L',S_R)$
		\IF {$|S_L'| > \sfrac{|S|}{2}$}
			\RETURN $S_L'$
		\ELSE
			\RETURN $\emptyset$
		\ENDIF
	\ELSIF {$S_L' = \emptyset$ and $S_R' \neq \emptyset$}
		\STATE $S_R' \gets appendEq(S_R',S_L)$
		\IF {$|S_R'| > \sfrac{|S|}{2}$}
			\RETURN $S_R'$
		\ELSE
			\RETURN $\emptyset$
		\ENDIF
	\ELSIF {$S_L' \neq \emptyset$ and $S_R' \neq \emptyset$}
		\STATE $S_L' \gets appendEq(S_L',S_R)$
		\IF {$|S_L'| > \sfrac{|S|}{2}$}
			\RETURN $S_L'$
		\ELSE
			\STATE $S_R' \gets appendEq(S_R',S_L)$
			\IF {$|S_R'| > \sfrac{|S|}{2}$}
				\RETURN $S_R'$
			\ELSE
				\RETURN $\emptyset$
			\ENDIF
		\ENDIF
	\ELSE
		\RETURN $\emptyset$
	\ENDIF
\ENDIF
\end{algorithmic}

$appendEq(S_1, S_2)$
\begin{algorithmic}[1]
\STATE $b \gets \text{any } s \in S_1$
\FOR {$a \in S_2$}
	\IF {$(a \sim b)$}
		\STATE $S_1 \gets S_1 \cup \{a\}$
	\ENDIF
\ENDFOR
\RETURN $S_1$
\end{algorithmic}

\textbf{5) Prove correctness of the algorithm using induction:}

Base Case: Calling $majElem(\{a\})$ results in the If statement in instruction
1 being satisfied, which causes instruction 2 to execute, which would return
$\{a\}$. Thus the base case holds.

Inductive Case:

Inductive Hypothesis - assume $majElem(S)$ is correct for all sets of size
greater than 1 to $n$. Consider the case where $|S| = n+1$:

Instruction 1 will not be satisfied, so instruction 4 will be executed - this
partitions $S$ into $S_L$ and $S_R$, where both $|S_L| \geq 1$ and $|S_R| \geq
1$. Thus both $|S_L| \leq n$ and $|S_R| \leq n$, so in instruction 5
$majElem(S_L)$ and $majElem(S_R)$ return correct output by the Inductive
Hypothesis. The If statements in instructions 6, 13, 20, and 32 branch into the
cases 1, 2, 3, and 4 (outlined in part 3 above) respectively, and implement the
algorithms described for each case; the correctness of these algorithms was also
proved in part 3 above, and so $majElem(S)$ returns correct output for case
$|S|=n+1$.

Thus correctness of $majElem(S)$ has been proved by induction. $\square$ 

\textbf{6) Find and prove a recurrence for cost of the algorithm:}

The recurrence in this case will not focus on the number of instructions
executed; rather, the object of concern is the number of equivalence tests
performed. The only point in the algorithm where an equivalence test is
performed is in the sub-function $appendEq(S_1,S_2)$, which is called from
instructions 7, 14, 21, and 25 in $majElem(S)$.

Within $appendEq(S_1,S_2)$, an equivalence test is performed on every element of
$S_2$ - thus it is performed $|S_2|$ times. If we look at every point
$appendEq(S_1,S_2)$ is called from $majElem(S)$, we see it is in all cases
called with the size of $S_2$ decided by instruction 4. The intention of
instruction 4 is to split $S$ into $S_L$ and $S_R$ - two sets of nearly equal
size, so the size of $S_2$ will be either $\lfloor \sfrac{|S|}{2} \rfloor$ or
$\lceil \sfrac{|S|}{2} \rceil$.

If the If statements on lines 20 and then 22 are not satisfied, we see that on a
single level of recursion $appendEq(S_1,S_2)$ will be called twice - first with
$S_2=S_R$ and then with $S_2=S_L$. Since $S_L$ and $S_R$ partition $S$, this
means exactly $|S|$ equivalence tests will be performed per level of recursion
in the worst case.

$majElem(S)$ recurses on both $S_L$ and $S_R$, which if $|S|$ is even are of
size exactly $\sfrac{|S|}{2}$. It suffices to consider only cases in which
$|S|$ is even, as when $|S|$ is odd the number of equivalence tests performed
will be bounded above by the number of tests performed on some set $S'$ where
$|S'| = |S|+1$. Coupled with the observation that in the base case no
equivalence tests are performed, we define the function $T(n)$, the number of
equivalence tests performed on input of size $n$:

\begin{displaymath}
T(n) \leq \left\{
	\begin{array}{lr}
	0 & : n=1 \\
	2T(\sfrac{n}{2}) + n & : n>1
	\end{array}
	\right.
\end{displaymath} 

\textbf{7) Use the recurrence to prove an upper bound on running time:}

The Master Theorem allows us to easily bound recurrences of the following form:
\begin{equation*}
T(n) = a \; T\!\left(\frac{n}{b}\right) + f(n)  \;\;\;\; \mbox{where} \;\; a
\geq 1 \mbox{, } b > 1
\end{equation*}
The recurrence we found is indeed of this form, so we may use the theorem.

In our recurrence, $a=2$, $b=2$, and $f(n) = n$. First, we must show that $f(n)
\in \Theta(n^{\log_ba})$:
\begin{align*}
f(n) &\in \Theta(n^{\log_ba}) \\
n &\in \Theta(n^{\log_22}) \\
n &\in \Theta(n)
\end{align*}
This is clearly true. Thus from the Master Theorem, we know the following:
\begin{align*}
E(n) &\in \Theta(n^{\log_ba}\log n) \\
E(n) &\in \Theta(n^{\log_22}\log n) \\
E(n) &\in \Theta(n \log n)
\end{align*}
So, $E(n) \in O(n \log n)$.

\section{Minimal Vertex Cover}

Building on the material covered in the handout on an efficient algorithm to
calculate the minimal vertex cover of a graph,$^{[1]}$ we may consider an
algorithm identical to the one outlined, but where specific conditions are
placed on the vertex $v$ selected. The vertex $v$ selected for an input graph
$G$ should be the vertex of highest order in that graph - that is, the vertex
touching the most edges.

If the highest-order vertex is of order one, then $G$ must consist only of
disjoint pairs of vertices (vertices of order one) and vertices without any
edges (vertices of order zero). $G$ would be very easy to find a minimal vertex
cover for in this special case (simply pick one vertex in each disjoint pair),
so the algorithm should detect this and correctly select the minimal vertex
cover. Given the ease of computation in this case, it will not be considered for
bounding the worst case above.

If the highest-order vertex is of order two, then $G$ must consist only of
$n\geq1$ disjoint subgraphs that are either a perfect cycle (so every vertex in
that subgraph is exactly of order two) or a chain of vertices (so every vertex
has order two except for the ends of the chain, which have order one). A minimal
vertex cover is also simple to compute in this case using a greedy algorithm,
which was produced in tutorial 6.

If the highest-order vertex is of order three, the problem becomes vastly more
complex. In this case, then, the algorithm defined in the minimal vertex cover
handout will be used. However, since we need not worry about cases where only
two or three vertices are removed when recursing the second time, we may
rewrite the recurrence thus:
\begin{displaymath}
T(n) \leq \left\{
	\begin{array}{lr}
	2 & : n=0\text{ or }n=1 \\
	T(n-1) + T(n-4) + cn^3 	& : n>1
	\end{array}
	\right.
\end{displaymath}
This has an associated characteristic polynomial:
\begin{equation*}
x^4 - x^3 - 1
\end{equation*}
This has only one real root - $\gamma \approx 1.38028$. Using the same reasoning
as in the handout, we arrive at the following:
\begin{equation*}
T(n) \in O(\gamma^n)
\end{equation*}
Which of course is equivalent to saying given a graph $G(V,E)$, the algorithm
runs in:
\begin{equation*}
O\left( \gamma^{|V|} \right)
\end{equation*}

\section{References}

\begin{enumerate}
\item Dr. Wayne Eberly, \textit{An Asymptotically Faster Algorithm for Vertex
Cover}. CPSC 413, Winter 2011
\end{enumerate}

\end{document}