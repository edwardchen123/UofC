\documentclass{article}
\usepackage{parskip}
\usepackage{amsmath}
\usepackage{amssymb}
\usepackage{cancel}

\setlength{\parindent}{0cm}

\begin{document}

\title{CPSC 413 \\ Assignment 2 - Greedy Algorithms}
\author{Andrew Helwer}
\date{February 2011}
\maketitle

\section{Show That a Feasible Output Always Exists}

Given a set of inputs $I=\{h_1,h_2,\ldots,h_n\}$ such that $0 \leq h_1 < h_2 <
\ldots < h_n$ where $h_1,h_2,\ldots,h_n \in \mathbb{R}$, show there exists a
set of outputs $O=\{b_1,b_2,\ldots,b_n\}$ such that $0 \leq b_1 < b_2 <
\ldots < b_n$ where $b_1,b_2,\ldots,b_n \in \mathbb{R}$ and $\forall \: h \in
I$, $\exists \: b \in O$ such that $|h - b| \leq 4$.

Consider the set of outputs $O$ where $\forall \: b_i \in O$ and $\forall \:
h_j \in I$ where $0 \leq i = j \leq n$, $b_i = h_j$. Then since $0 \leq h_1 <
h_2 < \ldots < h_n$, $0 \leq b_1 < b_2 < \ldots < b_n$, and since $b_i = h_j$,
$|h_j-b_i|=0$, so $|h_j-b_i| \leq 4$. Thus a feasible output exists for any
valid input. $\square$

In more visual terms, a tower is placed on top of every house. Then since no two
houses occupy the same position, no two towers occupy the same position. Since
every house has a tower on top of it, the distance from any house to the nearest
tower is zero.

\section{Show That a Correct Output Always Exists}

Define the output described in the first question to be the trivial solution. It
was demonstrated that for every valid input of length $n$, the trivial solution
is feasible and also of size $n$. So we can define an upper bound on the size
of correct output:

Given valid input of length $n$, assume that an output $o$ of length $m > n$
is correct. Then $o$ is as small as possible. But there exists the trivial
solution of length $n$, so $o$ is not as small as possible. Thus the size of
correct output may equal but not exceed the size of valid input.

The trivial solution is not always the correct solution. Consider the valid
input $\{1,2\}$ of length 2 and the feasible output $\{1\}$ of length 1. Since
an output of length less than the trivial solution exists, the trivial solution
is not correct.

Consider the set of all feasible outputs $F$ for some valid input of
finite length $n$, and then consider $O \subset F$ where $O = \{o \in F :
length(o) \leq n\}$. The trivial solution will be in $O$, so it will be
nonempty. We can then define the nonempty set $S = \{s \in \mathbb{N} : s =
length(o), o \in O\}$, which is composed of the unique lengths of $o \in O$.
Thus $S$ is a nonempty subset of $\mathbb{N}$, and so by the Least Number
Principle it has a
smallest element, $s$. Therefore the correct output will be any of the $o \in O$
such that $length(o) = s$, and so a correct output exists for any valid input.
$\square$

\section{Design and Prove Correctness of the Algorithm}

\textbf{Describe How to Solve Trivial Instances}

If the input set is empty, then the corresponding output set will also be empty
as there are no houses to provide coverage for. If the input set is $\{h\}$ of
size 1, then the correct output set will be $\{b\}$ where $h-4 \leq b \leq h+4$
and $0 \leq b$ - the size of the output set must be at least one, as otherwise
there will be no coverage at all, and a single tower may provide coverage for
the single house by being within range of it so it is the minimum and therefore
correct solution.

\textbf{Describe a Greedy Strategy}

Given an input set, $I$, and an output set, $O$, the greedy strategy is to
begin by picking the first (and therefore lowest) element of $I$, $h$, and set
$O := O \cup \{h+4\}$.

\textbf{Show That This Works - Correct Output Includes the Greedy Choice}

Consider an instance of the problem with some valid input, $I$, of length $n$,
where $n \geq 2$. Then there is a correct output of this instance that includes
$h_1+4$, the greedy choice.

Proof: 

By question two, a correct output exists for this problem instance.

Consider the given input set $I$. The first element of $I$ will be $h_1$, so
using the greedy strategy we set the (currently empty) output set $O := O \cup
\{h_1+4\}$. So since $|h_1 - (h_1+4)| \leq 4$, $h_1$ is covered.

Observe that any correct output for this instance must contain some tower that
covers $h_1$, by definition. Define this tower to be $b_1$. Since $\nexists \:
h_i \in I$ where $h_i \leq h_1$, no benefit is conferred by $b_1$ being at any
position other than $h_1+4$, as if $b_1 > h_1+4$ then $h_1$ will no longer be
covered by $b_1$, and if $b_1 < h_1+4$ it provides coverage to some interval
less than $h_1$, where no houses are located. Thus we can replace $b_1$ in the
correct solution with $h_1+4$ without rendering it incorrect (this also holds
if more than one tower covers $h_1$ - replacing either will do). Therefore there
is a correct output of this instance that includes the greedy choice. $\square$

Consider now the set $I':= \{h \in I : h > h_1+8\}$, which will
include all houses not covered by the greedy choice and is guaranteed to be of
size $\leq n-1$, as $h_1 \notin I'$. Then $I'$ is a valid input as it satisfies
the precondition, and so by the above the correct output for $I'$ will include
the greedy choice.

After making the greedy choice for $I'$, we may append it to $O$, and every
element of $O$ will be included in some correct output for $I$. Therefore we may
solve the problem recursively by making the greedy choice, removing all covered
houses, and making the greedy choice for the resulting subset until only the
empty set is left.

\textbf{Describe an Algorithm}

Given a valid input set $I$ satisfying the precondition, where $h_1$ is the
first element in $I$, and an empty set $O$:


\textbf{$cover(I)$}
\begin{enumerate}
\item if $I = \emptyset$ then

\item return $\emptyset$

else

\item $O := O \cup \{h_1 + 4\}$

For $h \in I$:

\item if $h \leq h_1+8$ then

\item $I:= I \setminus \{h\}$

else

\item break

\item return $O \cup cover(I)$
\end{enumerate}

\textbf{Proof of Correctness by Induction}

Note that the loop is guaranteed to halt as the input is of finite size.

Notation: $I$ is the input set satisfying the precondition, $h_1$ is the first
element of $I$

$P(n)$: $cover(I)$ is correct, for every $size(I) \in \mathbb{N}$

Base Case:

If $I = \emptyset$, then $cover(I) = \emptyset$ by instruction 1 evaluating to
True, which causes insstruction 2 to execute and return $\emptyset$. Then the
base case is correct, as no towers need to be placed if there are no houses to
be covered.

Strong Inductive Hypothesis:

Assume $P(n)$ to be correct for every input up to inputs of size $n$. Consider
the case of input of size $n+1$:

Upon execution of $cover(I)$ on $I$ where $size(I) = n+1$, $cover(I)$ returns
$\{h_1+4\} \cup cover(I')$, where $size(I') < size(I)$, and so $cover(I')$
returns the correct solution for $I'$ by the strong inductive hypothesis. As
was proved above, the union of the greedy choice for $I$ with a correct
solution for $I'$ will result in a correct solution for $I$. Thus $cover(I)$ is
correct, for every $size(I) \in \mathbb{N}$. $\square$

\section{Finding an Upper Bound on Running Time}

Consider the algorithm specified above run on valid input. Define $T(n)$ to
give the unit cost of running it on input of size $n$. Observe that if the size
of the input is 0 - that is, $I = \emptyset$, only instructions 1 and 2 will be
executed. Thus:

\begin{equation*}
T(0) = 2
\end{equation*}

For input of size 1, steps 1 and 3 will be executed, followed by the for
loop test 4 - given the first (only, in this case) $h_1 \in I$, $h_1 \leq
h_1+8$, so instruction 5 is then executed. Since there are no other elements of
$I$, 7 will be executed, where $I = \emptyset$, so we have:

\begin{equation*}
T(1) = 5 + T(0) = 7
\end{equation*}

The pattern is easy to see. Given an input set $I$ of size $n$, if $k \leq n$
elements are removed from $I$, then instruction 7 calls $cover(I)$ on the set
$I$ of size $n-k$. The loop to remove $k$ elements from $I$ results in the
execution of $2k+2$ instructions - just two instructions per element, plus
another two to break from the loop! Using all this information, we may define a
recurrence as follows, with $k$ some variable set by the nature of the input
$I$:

\begin{align*}
&T(0) = 2 \\
&T(n) = 3 + 2k + 2 + T(n-k)
\end{align*}

While correct, this is not very useful, as we wish $T(n)$ to be defined only in
terms of $n$. It would be best to find some way of removing $k$. Luckily, then,
when considering steps used by the algorithm in the worst case, it suffices to
consider only the case in which a single element is removed from $I$. By
setting $k=1$, we see that the amortized cost will be be 7 for a single level of
recursion - far larger than the paltry 2 instructions required during batch
removal from $I$. So, we may rewrite the recurrence thus:

\begin{align*}
&T(0) = 2 \\
&T(n) = 7 + T(n-1)
\end{align*} 

Now that we have a working recurrence, finding an upper bound is in order. This
is fairly simple - since $T(n-1)$ will be called on every level of recursion,
there will be $n+1$ levels in total with the $(n+1)$th being the base case,
which has cost 2. Since each non-base case level uses at least 7 instructions,
we may present the upper bound thusly:

\begin{equation*}
T(n) \leq 7n + 2
\end{equation*}

So, we have:

\begin{equation*}
T(n) \in O(n)
\end{equation*}

Just for fun, let us also find a lower bound on the running time - this would
occur when we set $k = n$, following from the same reasoning used to set $k=1$
to find the upper bound. Then $T(n) = 2 + 2k + 2 + T(n-k)$, or $T(n) = 2n + 4 +
T(0) = 2n + 6$. So, we have found $T(n) \geq 2n+6$.

From this, we have:

\begin{equation*}
T(n) \in \Theta(n)
\end{equation*}

\end{document}