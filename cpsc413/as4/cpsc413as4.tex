\documentclass{article}
\usepackage{parskip}
\usepackage{amsmath}
\usepackage{amssymb}
\usepackage{cancel}
\usepackage{algorithmic}

\setlength{\parindent}{0cm}

\begin{document}

\title{CPSC 413 \\ Assignment 4 - Dynamic Programming}
\author{Andrew Helwer}
\date{March 2011}
\maketitle

\section{Design a Divide and Conquer Algorithm}

Define a function $\omega:v \rightarrow \mathbb{R}$ which returns the real
weight of any vertex $v$.

Define a function $\alpha:S \rightarrow \mathbb{R}$, which for
any set of vertices $S$ has the following action:
\begin{equation*}
\alpha(S) = \sum_{v \in S} \omega(v)
\end{equation*}

Within this document, the term \textit{minimal vertex cover} will be synonymous
with a vertex cover $V \subset T$ such that for any vertex cover $V'
\subset T$, $\alpha(V) < \alpha(V')$.

$\{nil\} \cup S = S$ for any set $S$, where $nil$ is the value of $v.left$ or
$v.right$ if $v$ does not have a left or right child.

\textbf{1) Describe how to solve trivial instances:}

One trivial instance of this problem occurs when $|T| = 0$, that is when
$cover(nil)$ is called. $\emptyset$ should be returned in this instance.

The other trivial instance occurs when $|T| = 1$. $\emptyset$
will be returned, as a graph with a single vertex $v$ has no edges and thus the
vertex cover will either be $v$ or $\emptyset$. Since $\alpha(\{v\}) > 0$ and
$\alpha(\emptyset) = 0$, $\alpha(\emptyset) < \alpha(\{v\})$ and so $\emptyset$
is the vertex cover with minimal $\alpha$.

\textbf{2) Describe how sub-instances should be formed from nontrivial ones:}

Define the function $\pi(v)$ for any $v \in T$, where:
\begin{displaymath}
\pi(v) = \left\{
	\begin{array}{lr}
	\emptyset & : v = nil \\
	\{cover(v.left) \cup cover(v.right)\} & : v \neq nil
	\end{array}
	\right.
\end{displaymath} 
Consider the case of a tree $T$ of arbitrary size. Given the root of $T$, $v$,
split the problem as follows:
\begin{equation*}
S_1 := \{v\} \cup \pi(v)
\end{equation*}
\begin{equation*}
S_2 := \{v.left\} \cup \{v.right\} \cup \pi(v.left) \cup \pi(v.right)
\end{equation*}
This will find the minimal vertex cover that includes $v$, $S_1$, and the
minimal vertex cover that does not include $v$, $S_2$.

\textbf{3) Describe how a solution is formed from sub-instance solutions:}

The set of $V$ of all vertex covers for $T$ may be partitioned into two
subsets, where the root of $T$ is $v$:
\begin{itemize}
\item The set $V_1$ of all vertex covers for $T$ that include $v$
\item The set $V_2$ of all vertex covers for $T$ that do not include $v$
\end{itemize}
Since $V_1 \sqcup V_2 = V$, the minimal element of $V$ may be found by taking
the lesser of the minimal element in $V_1$ and the minimal element in $V_2$.
Since the minimal element in $V_1$ is $S_1$ and the minimal element in $V_2$ is
$S_2$, if $\alpha(S_1) \leq \alpha(S_2)$, the minimal element of $V$ will be
$S_1$. Otherwise, it will be $S_2$. So $cover(v)$ should return $S_1$ if
$\alpha(S_1) \leq \alpha(S_2)$ and $S_2$ otherwise.

\textbf{4) Describe a recursive algorithm implementation:}

$cover(v)$
\begin{algorithmic}[1]
\IF {($v.left=nil$ and $v.right=nil$) or $v=nil$}
	\RETURN $\emptyset$
\ELSE
	\STATE $S_1\gets\{v\}\cup\pi(v)$
	\STATE $S_2\gets\{v.left\}\cup\{v.right\}\cup\pi(v.left)\cup\pi(v.right)$
	\IF {$\alpha(S_1) \leq \alpha(S_2)$}
		\RETURN $S_1$
	\ELSE
		\RETURN $S_2$
	\ENDIF
\ENDIF
\end{algorithmic}

\textbf{5) Prove correctness of the algorithm using induction:}

Base Cases: $|S|=1$ and $|S|=0$

In both cases, the If statement in Instruction 1 will evaluate to True, and so
instruction 2 will be executed, returning $\emptyset$ as was shown to be
correct in Part 1. Thus the Base Cases hold.

Inductive Case:

Inductive Hypothesis - assume $cover(v)$ is correct for the root $v$ of all
trees $T$ of size greater than 1 to $n$. Consider the case where $|T|=n+1$, $v$
is the root of $T$, and $cover(v)$ is called:

The If statement in Instruction 1 will evaluate to False, and so Instructions 4
and 5 will be executed. These split the problem into sets $S_1$ and $S_2$ as
was defined in Part 2. Since all recursive calls of $cover(u)$ will have $u$ as
one of $v.left$, $v.right$, $v.left.left$, $v.left.right$, $v.right.left$, or
$v.right.right$, the tree for which $u$ is the root is guaranteed to be of size
less than or equal to $n$. Thus $cover(u)$ will return correct output by the
Inductive Hypothesis, and so the values of $S_1$ and $S_2$ will be the minimal
elements of $V_1$ and $V_2$. As was proved in Part 3, whichever of $S_1$ and
$S_2$ have the smaller value of $\alpha$ will be the minimal vertex cover of
$T$, which will be decided by Instructions 6, 7, and 9. Therefore $cover(v)$
returns correct output.

Thus the correctness of $cover(v)$ has been proved by induction. $\square$

\section{Design a Dynamic Programming Algorithm}

\textbf{6) Identify smaller instances that must be solved in nontrivial cases:}

Using the same procedure as in the Divide and Conquer algorithm, for a tree $T$
with root $v$, in order for $cover(v)$ to be solved we must know the minimal
vertex covers for the subtrees of $T$ for which the following are roots:
\begin{itemize}
\item $v.left$
\item $v.right$
\item $v.left.left$
\item $v.left.right$
\item $v.right.left$
\item $v.right.right$
\end{itemize}

\textbf{7) Identify an ordering on these smaller instances:}

Since we may represent instances of the problem as vertices in $T$ upon which
$cover(v)$ is called, for any $u,w \in T$, $u<w$ if and only if $u$ is visited
before $w$ during a Postorder Traversal of $T$. This traversal is defined using
the following function, with $v$ the root of $T$:

$postorder(v)$
\begin{algorithmic}[1]
\IF {$v.left \neq nil$}
	\STATE $postorder(v.left)$
\ENDIF
\IF {$v.right \neq nil$}
	\STATE $postorder(v.right)$
\ENDIF
\STATE visit $v$ and return
\end{algorithmic}

Claim 7.1: For any $v \in T$, if $postorder(v)$ recursively calls
$postorder(v.left)$ or $postorder(v.right)$, then $v.left$ and $v.right$ will
be visited before $v$.

Proof: This follows from the structure of the algorithm above; if
$postorder(v.left)$ is called from Instruction 2 inside $postorder(v)$, then no
other instructions in $postorder(v)$ may be executed until $postorder(v.left)$
returns in Instruction 7, so $v.left$ will have been visited. So $v.left$ is
visited before $v$. By symmetry, $v.right$ will be visited before $v$ as
well, and so Claim 7.1 is true. $\square$

Claim 7.2: For any vertex $v \in T$, all of the following are less than $v$ (if
they exist):
\begin{itemize}
\item $v.left$
\item $v.right$
\item $v.left.left$
\item $v.left.right$
\item $v.right.left$
\item $v.right.right$
\end{itemize}
Proof: Consider a call to $postorder(v)$. If $v.left$ exists,
$postorder(v.left)$ is immediately called, so $v.left$ is visited before $v$ by
Claim 7.1, and equivalently $v.left < v$. By symmetry, $v.right < v$ as
well. Inside the call to $postorder(v.left)$, if $v.left.left$ exists,
$postorder(v.left.left)$ is called, so $v.left.left$ is visited before $v.left$
by Claim 7.1, and equivalently $v.left.left < v.left$. Therefore by the
transitive property of an order relation, $v.left.left < v$. The proof is
symmetric for $v.left.right$, $v.right.left$, and $v.right.right$. Thus Claim
7.2 is true. $\square$

\textbf{8) Choose a data structure}

The data structure of choice will be a hash table, $H$, so that for any $v \in
T$, $H(v)$ returns the minimal vertex cover for the subtree of $T$ with $v$ as
its root in $O(1)$ time. $H(nil)$ will return $\emptyset$.

\textbf{9) Describe a dynamic programming implementation:}

First redefine the function $\pi(v)$ from above:
\begin{displaymath}
\pi(v) = \left\{
	\begin{array}{lr}
	\emptyset & : v = nil \\
	\{H(v.left) \cup H(v.right)\} & : v \neq nil
	\end{array}
	\right.
\end{displaymath}

$cover(v)$
\begin{algorithmic}[1]
\IF {$v.left \neq nil$}
	\STATE $cover(v.left)$
\ENDIF
\IF {$v.right \neq nil$}
	\STATE $cover(v.right)$
\ENDIF
\IF {$v.left = nil$ and $v.right = nil$}
	\STATE $H(v) \gets \emptyset$
\ELSE
	\STATE $S_1\gets\{v\}\cup\pi(v)$
	\STATE $S_2\gets\{v.left\}\cup\{v.right\}\cup\pi(v.left)\cup\pi(v.right)$
	\IF {$\alpha(S_1) \leq \alpha(S_2)$}
		\STATE $H(v) \gets S_1$
	\ELSE
		\STATE $H(v) \gets S_2$
	\ENDIF
\ENDIF
\RETURN $H(v)$
\end{algorithmic}

\textbf{10) Prove correctness of the algorithm using induction:}

Note that $cover(v)$ is simply a modified version of $postcover(v)$, with the
visit and return instruction replaced by a number of other instructions.

Base Case:

If $cover(v)$ is called where $v$ is the only vertex in its tree, then both
Instructions 1 and 4 will not be satisfied but Instruction 7 will be, so
Instruction 8 will be executed and $H(v)$ set to $\emptyset$, as it should be.
Thus the Base Case holds.

Inductive Step:

Inductive Hypothesis - assume $cover(v)$ is correct for the root $v$ of all
trees $T$ of size greater than 1 to $n$. Consider the case where $|T|=n+1$, $v$
is the root of $T$, and $cover(v)$ is called:

At least one of Instructions 1 and 4 will be satisfied, so $cover(v.left)$
or $cover(v.right)$ will be called, and since the subtree of $T$ for which
$v.left$ or $v.right$ is a root is guaranteed to be of size at most $n$, by the
Inductive Hypothesis $cover(v.left)$ and $cover(v.right)$ will be correct.
Instruction 7 will not be satisfied, so Instructions 10 and 11 will be executed.
By Claim 7.2, all of the vertices
required to form sets $S_1$ and $S_2$ will have been visited, and by the
Inductive Hypothesis their values in the hash table $K$ will be correct, so
the values of $S_1$ and $S_2$ will be the minimal elements of $V_1$ and $V_2$.
As
was proved in Part 3, whichever of $S_1$ and $S_2$ have the smaller value of
$\alpha$ will be the minimal vertex cover of $T$, which will be decided by
Instructions 12, 13, and 15. So $H(v)$ will contain the minimal vertex cover of
$T$, which is returned in Instruction 18. Therefore $cover(v)$ returns correct
output.

Thus the correctness of $cover(v)$ has been proved by induction. $\square$

\textbf{11) Define a bounding recurrence relation}

Analysis of $cover(v)$ will be simplified if we first perform some analysis of
$postorder(v)$. Observe that in the worst case, five instructions will be
executed on a single level of recursion in $postcover(v)$, two of which are
recursive calls. As this is a traversal, each vertex will be visited exactly
once, so Instruction 7 will be executed exactly $|V|$ times on different levels
of recursion. We may therefore bound the number of instructions executed by
$postorder(v)$ on a tree of size $n$ as follows:
\begin{equation*}
T(n) \leq 5n
\end{equation*}
In order to convert this equation into a form useful for analysis of
$cover(v)$, define the number of instructions necessary to visit a vertex as
the variable $c$. So:
\begin{equation*}
T(n) \leq 4n + cn
\end{equation*}
As was noted in the proof of correctness, $cover(v)$ is simply a modified
version of $postorder(v)$ whereby the visit instruction is replaced with a
number of operations. Looking at $cover(v)$, we see in the worst case there will
be six instructions executed when visiting a vertex, for example Instructions
7, 10, 11, 12, 15, and 18. So, we can set $c=6$ in the above equation to bound
the number of instructions executed by $cover(v)$ on a tree of size $n$:
\begin{equation*}
T(n) \leq 4n + 6n
\end{equation*}
\begin{equation*}
T(n) \leq 10n
\end{equation*}
Thus $T(n)$ is of order $n$, and so $cover(v)$ runs in $O(n)$ time.

\end{document}
