\documentclass{article}
\usepackage{parskip}
\usepackage{amsmath}

\setlength{\parindent}{0cm}

\begin{document}

\title{CPSC 413 \\ Assignment 1 - Analysis of Simple Algorithms}
\author{Andrew Helwer}
\date{February 2011}
\maketitle

\section{Partial Correctness of FindLeaf}

\emph{Precondition:}	
\begin{itemize}
\item A nonempty binary tree $T$ is given as input
\end{itemize}

\emph{Postcondition:}
\begin{itemize} \item A leaf of $T$ is returned as output 
\item The tree $T$ has not been changed 
\end{itemize}

\textbf{Assertion A:} $v:=$ the root of $T$ and $T$ has not been modified

\textbf{Assertion B:} $v:=$ a leaf of $T$ and $T$ has not been modified

\textbf{Assertion C:} $v$ is an interior vertex of $T$

\textbf{Assertion D:} The depth of $v$ has increased by one, and $T$ has not
been modified

Assuming the precondition is satisfied, $T$ is nonempty - that is, it contains
at least one element, the root vertex. $v$ is set equal to this vertex before
execution of the loop, satistfying assertion $A$ as simply changing the
reference $v$ does not modify $T$.

If there is only one vertex in $T$, then $v$ (the root vertex) is a leaf and
the loop invariant $C$ is not satisfied. If there is more than one vertex in
$T$, then $v$ will be an interior vertex of $T$ as the root must then have at
least one child (either left or right) and so $C$ will be satisfied and the
loop entered.

By satisfying $C$ and entering the loop, $v$ is guaranteed to have either a
left or right child, specified $v.left$ and $v.right$ respectively. Once the
loop body has executed, $v$ has been set to either $v.left$ or $v.right$, and so
the number of $v:=v.parent$ operations it would take to reach the root node has
increased by one - thus the depth of $v$ has increased by one, partially
satisfying $D$. As well, the operations $v:=v.left$ and $v:=v.right$ make no
modifications to $T$ - they change only the reference $v$. Thus $T$ is not
modified during the loop body, and $D$ is fully satisfied. Therefore $C
\Rightarrow D$ over execution of the loop body.

$C$ is once again tested for, and if satisfied the loop body again executes. If
$C$ is not satisfied, the loop is exited. Since $C$ is not satisfied, $v$ is not
an interior vertex of $T$. Since $v$ is not an interior vertex, by definition it
must be a leaf. Thus $B$ is partially satisfied, which in turn satisfies one of
the postrequisites as $v$ is returned. The second postrequisite, and second
part of $B$ - that $T$ has not been modified - is ensured by satisfying $D$ at
the end of every execution of the loop body, or $A$ if there is only one vertex
in $T$. Thus $A \Rightarrow B$ over execution of the findLeaf function.

The above is sufficient to prove that the satisfaction of the preconditions
prior to findLeaf execution implies the satisfaction of the postconditions
after findLeaf execution. Since termination has not been proven, findLeaf has
only been shown to be partially correct.

\section{Termination of FindLeaf}

The following assumes that a binary tree consists of a finite number of
vertices.

Assume that findLeaf is executed on valid input whereby the precondition is
satisfied. Observe that instructions 1 and 6 are guaranteed to execute only
once, as they are not contained within the body of a loop. They may therefore
be disregarded when proving termination, and only the loop (instructions 2
through 5) considered.

A fundamental property of binary trees is depth, which is the number of edges
on the longest path from the root to any leaf in $T$. The depth of the root
is therefore zero. For any interior vertex $v$ of $T$, setting $v:=v.left$ or
$v:=v.right$ increases the depth of $v$ by one, as it adds an additional edge
to traverse to reach the root.

In question 1 it was proved that after every execution of the loop
body on valid input, the depth of $v$ was increased by one. Define a new
function $D$, whereby if $D(v)$ is computed for some $v \in T$, $D$ returns the
depth of $v$ in $T$, and if $D(T)$ is computed for some binary tree $T$, $D$
returns the depth of $T$. Thus after every execution of the loop body, where
$v'$ is the value of $v$ before execution, $D(v'):=D(v')+1=D(v)$.

Define another function $B(T,v) = D(T) - D(v)$, called the bound function. Since
$D(v)$ increases by one every time the loop body executes and $D(T)$ remains
constant (due to the condition of $T$ not being modified), $|B|$ decreases in
value with every execution of the loop body. Because the size of $T$ is finite,
$D(T)$ is finite as well, and so the loop body will execute only a finite
number of times before $B(T,v) = 0$ - at this point, $D(v) = D(T)$, and so $v$
must be a leaf node of $T$, as otherwise $v$ possesses a child with depth
greater than the depth of the tree, by definition impossible. Since $v$ is not
an interior vertex of $T$, the loop invariant is not satisfied and so the loop
will terminate when $B(T,v) = 0$ after a finite number of executions.

Thus it has been shown that the loop body will execute only a finite number of
times on valid input, and so findLeaf will terminate on valid input.

\section{Upper Bound on Cost of FindLeaf}

As in question 2, observe that on valid input steps 1 and 6 are executed only
once, for a total constant cost of 2. Also observe that the depth of $T$ is
maximized when every vertex in $T$ counts toward the depth of $T$ - that is,
when $T$ is in a configuration resembling a linked list with each vertex having
exactly one parent and one child except for the root (which lacks a parent) and
the leaf (which lacks a child). It will be in this case then that the bound
function defined in question 2 is initialized to a value furthest from zero.

The configuration described above will result in $T$ having a depth of
$|T|-1$, where $|T|$ is the number of vertices in $T$, also known as the size of
$T$. Since the depth of the root node is zero, $B$ is initialized to the value
of $|T|-1$, which is also the number of times the loop runs as $B$
decreases by exactly one with every execution of the loop body.

Inside the loop, three instructions will be executed every time it is run - the
initial check to see if the loop invariant is satisfied, a check whether the
vertex has a left child, and the assignment of the vertex reference to either
its left or right child depending on the outcome of this check. Thus the number
of instructions executed before the loop terminates will be $3*n$, where $n$ is
the number of times the loop executes.

Assembling these observations, the worst case running time of findLeaf may be
given by $R_U(s) = 3(s-1)+2 = 3s-1$, where $s=|T|$.

\section{Lower Bound on Cost of FindLeaf}

Steps 1 and 6 are executed exactly once, resulting in a cost of 2.
Since the number of other instructions executed has been proven to
be a function of the depth of the input, finding the binary tree
configuration with the least depth per unit size will also result in
a function for the lower bound on instruction executions in findLeaf.

Logically, this configuration results when each depth level of $T$ is completely
filled before putting a vertex on the next level - the binary tree cannot be
compressed any further. The structure described is one that is as close to a
full binary tree as possible. Since the capacity of each level is double the
previous, the function that relates $|T|$ to $D(T)$ is given by
$D(|T|)=fl(\log_2(|T|))$, where $fl$ is the flooring function that transforms
any real number into the next lowest integer value.
Following the same derivations as in question 3, we arrive at the best case
running time of findLeaf given by $R_L(s)=3*fl(\log_2(s))+2$, where $s=|T|$.

\section{The Action of the VertexCover Function}

\textbf{Proof A:}

By the action of vertexCover instruction either 6 or 7, the parent of the root
of $T_L$ in $T$ is $w$. Due to instructions 12 or 13, there is no path leading
from any vertex in $T_U$ to $w$, and thus no path leading from any vertex in
$T_U$ to any vertex in $T_L$. Thus any vertex in $T_U$ is not a vertex in
$T_L$, and vice versa - they are disjoint sets, with the addition that $u$ and
$v$ are in neither.

Since $v$ is defined to be a leaf, it has no children. The children of $w$ are
$v$ and the root of $T_L$, which could be $null$. Thus the removal of $T_L \cup
\{v\} \cup \{w\}$ from $T$ results in no lost vertices - all removed vertices
are in $T_L \cup \{v\} \cup \{w\}$. $T_U$ may thus be written $T \setminus T_L
\cup \{v\} \cup \{w\}$. Therefore $T_U$, $T_L$, $\{w\}$, and $\{v\}$ partition
$T$.

From the above it follows that every vertex of $T$ except $v$ and $w$ is a
vertex of exactly one of $T_U$ or $T_L$, so that $|T| = |T_U| + |T_L| + 2$.

\textbf{Proof B:}

The basis for the proof of disjointness in part A was that after the execution
of instructions 12 or 13, there is no path leading from any vertex in $T_U$ to
$w$, and thus no path leading from any vertex in $T_U$ to any vertex in $T_L$.
This may be modified slightly to prove something similar for edges in $T$.
Observe that of all the edges in $T$, there are only three that join to at
least one element outside of $T_U$ and $T_L$ - the edge from $x$ to $w$ (before
removal), the edge from the root of $T_L$ to $w$, and the edge from $v$ to $w$.
If these edges are disregarded, due to the partitioning nature of $T_U$ and
$T_L$ and how edges by definition must join two vertices, all other edges must
exist within $T_U$ or $T_L$.

Thus it is true that every edge in $T$ is an edge in exactly one of $T_U$ or
$T_L$, except for the edges that touch the vertex $w$.

\section{Correctness of VertexCover}

Prove: For all nonnegative integers $n = |T|$, $vertexCover(T)$ returns the
correct output

Base cases:

If $|T|$ = 0 or $|T|$ = 1, then the vertex cover is the empty set, as
there are no edges in $T$ to connect to the vertices in $T$.

Strong Inductive Hypothesis:

Assume the correct vertex cover is returned for $0...n$.
Show $vertexCover(n+1)$ also holds:

$vertexCover(n+1)$ splits $T$ up such that $|T| = |T_U| + |T_L|
+ 2$. Thus $|T_U| + |T_L| = |T|-2$, and so $|T_U| < |T|-2$ and $|T_U| < |T|-2$,
so by inductive hypothesis $vertexCover(T_U)$ and $vertexCover(T_L)$ return the
correct vertex cover of $T_U$ and $T_L$ respectively. Since $vertexCover(n+1) =
vertexCover(T_U) \cup vertexCover(T_L) \cup \{w\}$, $vertexCover(n+1)$ also
returns the correct vertexCover and so by the strong inductive hypothesis
vertexCover is correct for all nonnevative integers $n = |T|$.

\section{Worst-Case Cost of VertexCover}

The worst case running time will result from the same circumstances as the
worst case for findLeaf - that of a linked list configuration for $T$. Since
this is a recursive function, it is best to consider instruction 15 and work
from there.

Step 15 specifies returning $vertexCover(T_U) \cup vertexCover(T_L) \cup
\{w\}$. In the linked list configuration, $T_L$ will always be $null$, leading
to a cost of 2 for calling $vertexCover(T_L)$ as only instructions 1 and 2 will
be executed. The real cost lies in calling $vertexCover(T_U)$ - observe that by
removing both the leaf node and its parent from $T$ upon every execution of the
body of vertexCover, $|T_U| = |T|-2$. Since the precondition does not specify
whether $|T|$ is even or odd, this necessitates two base cases:

\begin{enumerate}
\item if $s = 0$, $C(s) = 2$
\item if $s = 1$, $C(s) = 2$
\item otherwise $C(s) = R_U(s) + 11 + 2 + C(s-2)$
\end{enumerate}

In the above, $s$ is initialized to $|T|$, $R_U(s)$ is the cost function of the
worst case for findLeaf defined back in question 3, and 11 is then number of
instructions that will be executed during a single level of recursion in
vertexCover, with 2 the number of instructions executed during the
$vertexCover(T_L)$ call.

\section{Upper Bound on Cost of VertexCover}

If we consider the recurrence defined in the previous question, it's clear that
there will be approximately $\frac{|T|}{2}$ levels of recursion. Consider the
quantities $A = C(s)$ and $B = C(s-2)$. As the recurrence is defined, $A$ must
be greater than or equal to $B$. Thus $A' = C(s) - C(s-2)$ is greater than or
equal to $B' = C(s-2) - C(s-4)$, and so $\frac{|T|}{2}(C(s) - C(s-2)) \geq C(s)$
for some large value of $s$, as $\frac{|T|}{2}(C(s) - C(s-2))$ may be viewed as
replacing the number of operations performed by all $\frac{|T|}{2}$ levels of
recursion with $C(s) - C(s-2)$, an increase over what they would have been.

Observing the above, the upper bound of the recurrence may be written
$\frac{|T|}{2}(R_U(s) + 13)$, which is of course linear in the product of the
size of $T$ and the number of steps used by findLeaf in the worst case.

\section{Lower Bound on Cost of VertexCover}

Observing that the best-case running time will occur if $T$ is as close
to a full binary tree as possible, using similar arguments to question 8 it is
ascertained that the lower bound on the running time will be of the order
$\frac{|T|}{2}(R_L(s) + C)$, where $C$ is some positive integer constant and
$R_L(s)$ is the lower bound on the running time of findLeaf defined in question
4. Thus the lower bound is also linear in the product of the size of $T$ and
the number of steps used by findLeaf in the best case.

\section{Bounds of VertexCover in Asymptotic Notation}

The worst-case running time of vertexCover, as a function of the size of the
input tree, $n$, is:

\begin{equation}
f(n) \in O(n^2)
\end{equation}

\end{document}