\documentclass{article}
\usepackage{parskip}
\usepackage{amsmath}
\usepackage{amssymb}
\usepackage{listings}

\setlength{\parindent}{0cm}

\begin{document}

\lstset{language=Haskell, frame=single}

\title{CPSC 449 Assignment 3 \\ J. Gallagher T01 MW 1400-1450}
\author{Andrew Helwer}
\date{Winter 2011}
\maketitle

\section{}

\textbf{Question 1a}

A Haskell algebraic type to represent well-formed formulas:

\begin{lstlisting}
data Formula 	= Symb Char |
 		Not Formula |
 		And Formula Formula |
 		Or Formula Formula
\end{lstlisting}

\textbf{Question 1b}

A Haskell expression that constructs a Formula represenation of $\neg(\neg p
\wedge \neg q)$:

\begin{lstlisting}
Not (And (Not (Symb 'p')) (Not (Symb 'q')))
\end{lstlisting}

\textbf{Question 1c}

Name: showFormula

Purpose: to return a string representation of a formula

Parameters: an expression of type Formula as defined above

Preconditions: the expression is a valid expression of type Formula

Returns: a string representation of the Formula

\begin{lstlisting}
showFormula :: Formula -> String
showFormula (Symb a)	= [a]
showFormula (Not f)	= "~"++(showFormula f)
showFormula (And f1 f2)	= "("++(showFormula f1)++"^"++
				(showFormula f2)++")"
showFormula (Or f1 f2)	= "("++(showFormula f1)++"v"++
				(showFormula f2)++")"
\end{lstlisting}

Make Formula an instance of Show class to easily print output:

\begin{lstlisting}
instance Show Formula where
    show f = (showFormula f)
\end{lstlisting}

\textbf{Question 1d}

Name: rewrite

Purpose: to remove all disjunctions from a Formula without altering meaning

Parameters: an expression of type Formula as defined above

Preconditions: the expression is a valid expression of type Formula

Returns: a Formula logically equivalent to input without any disjunctions

\begin{lstlisting}
rewrite :: Formula -> Formula
rewrite (Symb a)	= (Symb a)
rewrite (Not f)		= (Not (rewrite f))
rewrite (And f1 f2)	= (And (rewrite f1) (rewrite f2))
rewrite (Or f1 f2)	= (Not (And (Not (rewrite f1)) 
				(Not (rewrite f2))))
\end{lstlisting}

\textbf{Question 1e}

Principle of Structural Induction for Formula:

In order to prove some property P(f) $\forall$ finite formula f, it suffices
to prove the following hold:

\begin{itemize}
\item Base case: P(Symb a) $\forall$ Char a
\item Not case: If P(f) then P(Not f) $\forall$ finite formula f
\item And case: If P(f1) and P(f2) then P(And f1 f2) $\forall$ finite formulas
f1, f2
\item Or case: If P(f1) and P(f2) then P(Or f1 f2) $\forall$ finite formulas f1,
f2
\end{itemize}

\section{}

P(f): (rewrite f) = rewrite (rewrite f) $\forall$ finite formula f.

Prove P(f) using the Principle of Structural Induction for Formula.

\begin{itemize}

\item Base case: P(Symb a) $\forall$ Char a

Show rewrite (rewrite (Symb a)) = rewrite (Symb a):

\begin{align*}
&\text{rewrite (Symb a)} \\
&\rightsquigarrow \text{ (Symb a)}		&\text{(rewrite.1)}
\end{align*}

\begin{align*}
&\text{rewrite (rewrite (Symb a))} \\
&\rightsquigarrow \text{  rewrite (Symb a)}		&\text{(rewrite.1)} \\
&\rightsquigarrow \text{ (Symb a)}				&\text{(rewrite.1)}
\end{align*}

Thus the Base case holds.

\item Not case: If P(f) then P(Not f) $\forall$ finite formula f

Show rewrite (rewrite (Not f)) = rewrite (Not f):

Inductive Hypothesis:

Assume P(f): rewrite (rewrite f) = rewrite f

\begin{align*}
&\text{rewrite (Not f)} \\
&\rightsquigarrow \text{ Not (rewrite f)}		&\text{(rewrite.2)}
\end{align*}

\begin{align*}
&\text{rewrite (rewrite (Not f))} \\
&\rightsquigarrow \text{ rewrite (Not (rewrite f))}		&\text{(rewrite.2)} \\
&\rightsquigarrow \text{ Not (rewrite (rewrite f))}		&\text{(rewrite.2)} \\
&\rightsquigarrow \text{ Not (rewrite f)}				&\text{P(f)}
\end{align*}

Thus the Not case holds.

\item And case: If P(f1) and P(f2) then P(And f1 f2) $\forall$ finite formulas
f1, f2

Show rewrite (rewrite (And f1 f2)) = rewrite (And f1 f2):

Inductive Hypothesis:

Assume P(f1): rewrite (rewrite f1) = rewrite f1

Assume P(f2): rewrite (rewrite f2) = rewrite f2

\begin{align*}
&\text{rewrite (And f1 f2)} \\
&\rightsquigarrow \text{ And (rewrite f1) (rewrite f2)}		&\text{(rewrite.3)}
\end{align*}

\begin{align*}
&\text{rewrite (rewrite (And f1 f2))} \\
&\rightsquigarrow \text{ rewrite (And (rewrite f1) (rewrite f2))}
&\text{(rewrite.3)} \\
&\rightsquigarrow \text{ And (rewrite (rewrite f1)) (rewrite (rewrite f2))}
&\text{(rewrite.3)} \\
&\rightsquigarrow \text{ And (rewrite f1) (rewrite f2)}
&\text{(P(f1),P(f2))}
\end{align*}

Thus the And case holds.

\item Or case: If P(f1) and P(f2) then P(Or f1 f2) $\forall$ finite formulas
f1, f2

Show rewrite (rewrite (Or f1 f2)) = rewrite (Or f1 f2):

Inductive Hypothesis:

Assume P(f1): rewrite (rewrite f1) = rewrite f1

Assume P(f2): rewrite (rewrite f2) = rewrite f2

\begin{align*}
&\text{rewrite (Or f1 f2)} \\
&\rightsquigarrow \text{ Not ( And (Not (rewrite f1)) (Not (rewrite f2)))}	
&\text{(rewrite.4)}
\end{align*}

\begin{align*}
&\text{rewrite (rewrite (Or f1 f2))} \\
&\rightsquigarrow \text{ rewrite (Not ( And (Not (rewrite f1)) (Not (rewrite
f2))))}	&\text{(rewrite.4)} \\
&\rightsquigarrow \text{ Not (rewrite (And (Not (rewrite f1)) (Not (rewrite
f2))))}	
&\text{(rewrite.2)} \\
&\rightsquigarrow \text{ Not (And (rewrite (Not (rewrite f1))) (rewrite (Not
(rewrite f2))))}	&\text{(rewrite.3)} \\
&\rightsquigarrow \text{ Not (And (Not (rewrite (rewrite f1))) (Not
(rewrite (rewrite f2))))}	&\text{(rewrite.2)} \\
&\rightsquigarrow \text{ Not ( And (Not (rewrite f1)) (Not (rewrite f2)))}	
&\text{(P(f1),P(f2))}
\end{align*}

Thus the Or case holds.

By the Principle of Structural Induction for Formula, P(f) holds $\forall$
finite formula f. Thus (rewrite f) = rewrite (rewrite f) $\forall$ finite
formula f. $\square$

\end{itemize}

\section{}

\textbf{Question 3a}

Name: or

Purpose: to determine whether at least one element of the list is True

Parameters: a list of Bools

Preconditions: the input is a valid list of Bools

Returns: True if at least one element of the list is True, False otherwise

\begin{lstlisting}
or :: [Bool] -> Bool
or xs = foldr (||) False xs
\end{lstlisting}

\textbf{Question 3b}

Name: count

Purpose: to determine the number of occurences of an Int in a list of Ints

Parameters: an Int to be checked and a list of Ints

Preconditions: the input is a valid list of Ints and an Int

Returns: an Int representing the number of occurences of the element in the list

\begin{lstlisting}
count :: Int -> [Int] -> Int
count x ys = length (filter (==x) ys)
\end{lstlisting}

\section{}

\textbf{Question 4a}

Name: iter

Purpose: to apply f n times to some input x

Parameters: an Int representing number of composites, a function, and an input

Preconditions: the function is of type (a\textemdash$>$a) and the input is of
type a

Returns: output of type a resulting from applying f to x n times

The type of iter will be a, where a is the type of the input to the function,
which has type (a\textemdash$>$a). The function works by first generating a
list composed solely of the function f of length n using replicate, then folding
the function composition operator (.) into the list, ending with the identity
function id. The composition of f's is then applied to x.

\begin{lstlisting}
iter :: Int->(a->a)->a->a
iter n f x = (foldr (.) id (replicate n f)) x
\end{lstlisting}

\textbf{Question 4b}

Name: double

Purpose: given some integer, double multiplies it by two

Parameters: some Int

Preconditions: none

Returns: double the input

\begin{lstlisting}
double :: Int -> Int
double x = 2*x
\end{lstlisting}

Name: twoToThePowerOf

Purpose: to calculate the nth power of two

Parameters: some Int

Preconditions: input is greater than or equal to 0

Returns: two raised to the power of the input

\begin{lstlisting}
twoToThePowerOf :: Int -> Int
twoToThePowerOf n = (iter n (double) 1)
\end{lstlisting}

\end{document}