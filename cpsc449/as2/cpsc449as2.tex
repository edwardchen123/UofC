\documentclass{article}
\usepackage{parskip}
\usepackage{amsmath}
\usepackage{amssymb}
\usepackage{listings}

\setlength{\parindent}{0cm}

\begin{document}

\lstset{language=Haskell, frame=single}

\title{CPSC 449 Assignment 2 \\ J. Gallagher T01 MW 1400-1450}
\author{Andrew Helwer}
\date{Winter 2011}
\maketitle

\thispagestyle{myheadings}

\section{Programming: Questions 1-3}

\textbf{Question 1a}

Purpose: to square all odd items in a list of integers

Parameters: a list of integers

Preconditions: none

Returns: a list of integers

\begin{lstlisting}
oddSquaresA :: [Int]->[Int]
oddSquaresA xs = [x*x | x<-xs, (odd x)]
\end{lstlisting}

\textbf{Question 1b}

Purpose: to square all odd items in a list of integers

Parameters: a list of integers

Preconditions: none

Returns: a list of integers

\begin{lstlisting}
oddSquaresB :: [Int]->[Int]
oddSquaresB [] = []
oddSquaresB (x:xs)
    | (odd x)	= (x*x):(oddSquaresB xs)
    | otherwise	= (oddSquaresB xs)
\end{lstlisting}
    
\newpage

\thispagestyle{myheadings}
\markright{10023875}

\textbf{Question 2 (List Comprehension)}

Purpose: to remove all duplicates from a set of integers

Parameters: a list of integers

Preconditions: none

Returns: a list of unique integers

\begin{lstlisting}
uniqueA :: [Int]->[Int]
uniqueA xs = [x | x<-xs, (isUniqueIn x xs)]
\end{lstlisting}

\textbf{Question 2 (Primitive Recursion)}

Purpose: to remove all duplicates from a set of integers

Parameters: a list of integers

Preconditions: none

Returns: a list of unique integers

\begin{lstlisting}
uniqueB ::[Int]->[Int]
uniqueB [] = []
uniqueB (x:xs)
    | (contains x xs)	= (uniqueB (removeAll x xs))
    | otherwise		= x:(uniqueB xs)
\end{lstlisting}
    
\textbf{Question 3a}

Purpose: to merge two sorted lists such that the end result is sorted

Parameters: two lists of integers to be merged

Preconditions: both lists of integers are sorted in ascending order

Returns: a sorted list of integers

\begin{lstlisting}
mergeLists :: [Int]->[Int]->[Int]
mergeLists [] [] = []
mergeLists [] ys = ys
mergeLists xs [] = xs
mergeLists (x:xs) (y:ys)
    | x<=y		= x:(mergeLists xs (y:ys))
    | otherwise		= y:(mergeLists (x:xs) ys)
\end{lstlisting}
    
\newpage

\pagestyle{myheadings}

\textbf{Question 3b}

Purpose: to split a list into two lists, one of even indices one of odd

Parameters: a list of integers to be split

Preconditions: none

Returns: a tuple of two integer lists whose union is the original list

\begin{lstlisting}
splitList :: [Int]->([Int],[Int])
splitList [] = ([],[])
splitList (x:xs)
    | (xs==[])	= ([x],[])
    | otherwise	= (x:(skipOne (tail xs)),(skipOne xs))
\end{lstlisting}

Purpose: to generate a list from an input list by taking every second element

Parameters: a list of integers

Preconditions: none

Returns: a list of integers

\begin{lstlisting}
skipOne :: [Int]->[Int]
skipOne [] = []
skipOne (x:xs)
    | (xs==[])		= [x]
    | otherwise		= x:(skipOne (tail xs))
\end{lstlisting}

\textbf{Question 3c}

Purpose: to sort a list of integers with mergesort

Parameters: a list of integers

Preconditions: none

Returns: a list of integers

\begin{lstlisting}
mSort :: [Int]->[Int]
mSort [] = []
mSort (x:xs)
    | (xs==[])	= [x]
    | otherwise	= mergeLists(mSort(x:(skipOne(tail xs))))
				    (mSort(skipOne xs))
\end{lstlisting}
    
\newpage

\textbf{Useful functions written by myself:}

Purpose: to check if a list of integers contains a certain element

Parameters: an element to be checked and a list of integers

Preconditions: none

Returns: True or False if the element is or is not in the list

\begin{lstlisting}
contains :: Int->[Int]->Bool
contains x xs
    = case xs of
	   [] -> False
	   (y:ys) -> (x==y) || (contains x ys)
\end{lstlisting}

Purpose: to find the number of instances of an element in a list

Parameters: an element to be checked and a list of integers

Preconditions: none

Returns: an integer specifying number of occurences in the list

\begin{lstlisting}
instancesOf :: Int->[Int]->Int
instancesOf n [] = 0
instancesOf n (x:xs)
    | (n==x)		= 1 + (instancesOf n xs)
    | otherwise		= (instancesOf n xs)
\end{lstlisting}

Purpose: to determine if there is exactly one occurence of an element in a list

Parameters: an element to be checked and a list of integers

Preconditions: none

Returns: True or False if the element is or is not unique

\begin{lstlisting}
isUniqueIn :: Int->[Int]->Bool
isUniqueIn x xs = (instancesOf x xs) == 1
\end{lstlisting}
    
\newpage

Purpose: to remove all instances of an element in a list

Parameters: an element to be removed and a list of integers

Preconditions: none

Returns: a list with all occurences of the specified element gone

\begin{lstlisting}
removeAll :: Int->[Int]->[Int]
removeAll n [] = []
removeAll n (x:xs)
    | (x==n)		= (removeAll n xs)
    | otherwise		= x:(removeAll n xs)
\end{lstlisting}

\section{Written Proof: Question 4}

\textbf{(a)}

Base Case: $P([\:])$

Induction Step: $P(xs) \Rightarrow P(x:xs)$

\textbf{(b)}

Proving the base case $P([\:])$, $\forall \: n \in \mathbb{Z}$:

\begin{align*}
&\text{ or (match n [ ])} \\
&\rightsquigarrow \text{or ([ ])}			&(match.1) \\
&\rightsquigarrow \text{False}				&(or.1)
\end{align*}

Thus \textbf{or (match x [ ]) = False}.

\begin{align*}
&\text{ elem n [ ]} \\
&\rightsquigarrow \text{False}				&(elem.1)
\end{align*}

Thus \textbf{elem x [ ] = False} as well, and so the base case holds.

\newpage

\textbf{(c)}

Proving the inductive step, $P(xs) \Rightarrow P(x:xs)$:

Assume $P(xs)$ is true. Then \textbf{or (match n xs) = elem n xs}, $\forall \:
n \in \mathbb{Z}$.

Consider $P(x:xs)$:

\begin{align*}
&\text{ or (match n (x:xs))} \\
&\rightsquigarrow \text{ or ((n==x):(match n xs))}	&(match.2) \\
&\rightsquigarrow \text{ (n==x)$||$(or (match n xs))}	&(or.2)		
\end{align*}

Thus \textbf{or (match n (x:xs)) = (n==x)$||$(or (match n xs))}.

\begin{align*}
&\text{ elem n (x:xs)} \\
&\rightsquigarrow \text{ (n==x)$||$(elem n xs)}		&(elem.2) \\
\end{align*}

Thus \textbf{elem n (x:xs) = (n==x)$||$(elem n xs)}. So we have:

\begin{align*}
&\text{elem n (x:xs) = (n==x)$||$(elem n xs)} \\
&\text{elem n (x:xs) = (n==x)$||$(or (match n xs))}	&\text{by }P(xs) \\
&\text{elem n (x:xs) = or (match n (x:xs))} \\
\end{align*}

So $P(xs) \Rightarrow P(x:xs)$, and by structural induction the equality
\textbf{or (match n xs) = elem n xs} holds $\forall \: n \in \mathbb{Z}$ and
$\forall$ finite lists xs. $\square$

\end{document}