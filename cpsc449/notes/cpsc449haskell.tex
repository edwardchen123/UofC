\documentclass{article}
\usepackage{parskip}
\usepackage{amsmath}
\usepackage{amssymb}
\usepackage{listings}

\setlength{\parindent}{0cm}

\begin{document}

\lstset{language=Haskell, frame = single}

\title{CPSC 449 - Haskell}
\author{Andrew Helwer}
\date{Winter 2011}
\maketitle

\section{Programming Paradigms}

\begin{itemize}
\item A programming paradigm determines how you will solve a given problem
\item Imperative programming:
\begin{itemize}
\item Has a global, mutable state
\item Commands mutate the global state in some way
\item A program is a sequence of commands
\item Repetition is achieved through iteration
\item Example languages: Pascal, Fortran, Algol, C, ADA, Python
\end{itemize}
\item Object-oriented programming:
\begin{itemize}
\item Imperative programming with data abstraction
\item Example languages: C++, Java, Smalltalk, C$\#$, Python
\end{itemize}
\item Functional programming:
\begin{itemize}
\item Uses evaluation of expressions instead of execution of commands
\item No global state
\item Repetition is achieved through recursion
\item Example langauges: Haskell, ML, Scheme, Lisp
\end{itemize}
\item Logic programming:
\begin{itemize}
\item Describes result in terms of attributes and constraints
\item Complete opposite of imperative programming
\item Example languages: Prolog, G\"{o}del
\end{itemize}
\end{itemize}

\section{Haskell Basics}

\begin{itemize}
\item Function names start with a lowercase letter, types with an uppercase:
\begin{lstlisting}
name :: Type
\end{lstlisting}
\item Multiple parameters may also be defined (Type3 is the returned type):
\begin{lstlisting}
name :: Type1->Type2->Type3
\end{lstlisting}
\item Functions are implemented as follows:
\begin{lstlisting}
name args = body
\end{lstlisting}
\item Pattern matching is one way of enabling conditional function behaviour:
\begin{lstlisting}
name valA = valB
name valC = valD
\end{lstlisting}
\item Pattern matching is checked sequentially and returns if a match is found
\item Underscore (\_) matches arguments not used in the return value expression
\item To specify function behaviour for a collection of values, guards are used:
\begin{lstlisting}
name arg1, arg2
	| (f arg1 arg2)	= arg1
	| otherwise		= arg2
\end{lstlisting}
\item The standard library in Haskell is known as Prelude
\item Modules may be defined by placing the following in a Haskell file:
\begin{lstlisting}
module ModuleName where
\end{lstlisting}
\item Modules may also be imported into other modules:
\begin{lstlisting}
import ModuleName
\end{lstlisting}
\item It is sometimes necessary to override a Prelude function with your own:
\begin{lstlisting}
import Prelude hiding (name)
\end{lstlisting}
\item Logical operators on Bools:
\begin{itemize}
\item And: $\&\&$
\item Or: $||$
\item Not: not
\end{itemize}
\item Ordering operators on Ints:
\begin{itemize}
\item Less than: $<$
\item Greater than: $>$
\item Equal to: $==$
\item Not equal to: $/=$
\item Less than or equal to: $<=$
\item Greater than or equal to: $>=$
\end{itemize}
\end{itemize}

\section{Evaluation of Expressions}

\begin{itemize}
\item Every subexpression of an expression is called a redex
\item Every evaluation step reduces a redex to a simpler but equivalent form
\item Haskell has lazy evaluation, so expressions are expanded before reduction:
\begin{align*}
&23 - (\text{double }(3+1)) \\
&\rightsquigarrow 23 - (2*(3+1))            & \text{double} \\
&\rightsquigarrow 23 - (2*4)                & \text{arithmetic} \\
&\rightsquigarrow 23 - 8                    & \text{arithmetic} \\
&\rightsquigarrow 15                        & \text{arithmetic} 
\end{align*}
\item In the above, double was evaluated on (3+1) before (3+1) was evaluated
\item A slightly different process is followed to evaluate functions with
guards:
\begin{align*}
&\text{maxThree  6 (4+3) 5} \\
&\text{?? } 6 >= (4+3) \text{ \&\& } 6 >= 5 		&\text{check maxThree.1}\\
&??\rightsquigarrow 6 >= 7 \text{ \&\& } 6 >= 7 	&\text{logic} \\
&??\rightsquigarrow \text{False } \&\& \text{ True} &\text{logic} \\
&??\rightsquigarrow \text{False } 					&\text{logic} \\
&\text{?? } 7 >= 5 									&\text{check maxThree.2} \\
&??\rightsquigarrow \text{True } 					&\text{logic}	\\		
&\rightsquigarrow 7									&\text{maxThree.2} \\
\end{align*}
\end{itemize}

\section{Tuples and Lists}

\begin{itemize}
\item Tuples are composed of $n$ values of certain (possibly different) types:
\begin{lstlisting}
type Tuple = (Type1,Type2,Type3)
\end{lstlisting}
\item Elements in a pair tuple may be accessed using fst and snd
\item Lists are composed of an arbitrary number of values of a single type
\item The list constructor [ ] is an element of every list type
\item The operator ++ is used to concatenate two lists together ([a]++[b])
\item The operator : is used to prepend some element to a list (a:[b])
\item The operator !! returns the nth element of the list (xs!!n)
\item Prelude list functions:
\begin{itemize}
\item concat: concatenates a list of lists into a single list
\item length: returns the number of elements in the list
\item head, last: returns the first or last element of the list
\item tail, init: returns all but first or last elements of the list
\item take, drop: returns or removes the first n elements of the list
\item splitAt: splits the list into a pair tuple of lists at the nth element
\item reverse: reverses the order of the elements in the list
\item zip, unzip: turns a pair of lists into a list of pairs or vice versa
\item and, or: returns the conjunction or disjunction of a list of booleans
\item sum, product: returns the sum or product of a list of numerics
\end{itemize}
\item List comprehension generates a list from some other list xs:
\begin{itemize}
\begin{lstlisting}
[(f n) | n<-xs, (g n)]
\end{lstlisting}
\item Generation: n $<$\textemdash xs generates all elements of xs
\item Test: (g n) tests every $n$ in xs for a certain condition
\item Transform: (f n) transforms every $n$ in xs such that (g n) is true
\end{itemize}
\end{itemize}

\section{Recursion in Haskell}

\begin{itemize}
\item Instead of iteration, repetition is achieved through recursion in Haskell
\item Primitive recursion:
\begin{itemize}
\item Composed of two cases - base case and recursive case
\item Recursive case calls function with smaller input
\item Termination occurs when base case is reached
\item Primitive recursion is only applicable when using greedy algorithms
\end{itemize}
\item There are three fundamental patterns of primitive recursion:
\begin{itemize}
\item Accumulation: combine elements of a list using some operation
\item Transformation: map every element of a list to a different value
\item Selection: generate a subset of a list according to some restriction
\end{itemize}
\item General recursion:
\begin{itemize}
\item May have more than one base case
\item May recurse on multiple arguments
\item More difficult to prove correctness and termination
\item Any algorithm may be implemented using general recursion
\item Every primitive recursive function is a general recursive function
\end{itemize}
\end{itemize}

\section{Defining Functions Over Lists}

\begin{itemize}
\item In actuality lists are defined $x_1:x_2:\ldots:x_n:[\:]$
\item This easily allows primitive recursion to be used on lists
\item Patterns are one of the following:
\begin{itemize}
\item Literal value: True, 0, etc.
\item Variable: x, xs, etc.
\item Wildcard: (\_)
\item Tuple pattern: $(p_1,\ldots,p_n)$ where $p_1,\ldots,p_n$ are patterns
\item Constructor applied to patterns: (x:xs)
\end{itemize}
\item Pattern matching is used to decide which operations to perform on a list:
\begin{lstlisting}
name :: [Type]->[Type]
name [] 		= []
name (x:xs)
	| (f x)		= x:(name xs)
	| otherwise 	= (name xs)
\end{lstlisting}
\item The list constructor [ ] is the base case, with : used in the recursive
case
\item The case construction may also be used:
\begin{lstlisting}
name :: [Type]->[Type]
name xs
    = case (f xs) of
	[] -> []
	(y:ys) -> y:(name ys)
\end{lstlisting}
\item Local definitions may be created inside a function to increase simplicity:
\begin{lstlisting}
name :: [Type]->[Type]
name xs	= (f ys zs)
	where
	ys = (g xs)
	zs = (h xs)
\end{lstlisting}
\end{itemize}

\section{Structural Induction}

\begin{itemize}
\item Proving total correctness requires proving the following:
\begin{itemize}
\item Partial correctness: assuming termination, output is correct
\item Termination: the function terminates on all valid input
\end{itemize}
\item Termination is guaranteed with primitive recursive functions
\item Structural induction thus proves total correctness
\item Structural induction - show $P(ys)$ holds for all finite lists $ys$:
\begin{itemize}
\item Show the base case $P([\:])$ holds
\item Show the inductive step $P(xs) \Rightarrow P(x:xs)$ holds
\end{itemize}
\item Strengthening the inductive hypothesis is a good strategy in some cases
\item Note that structural induction is distinct from mathematical induction
\end{itemize}

\section{Higher-Order Functions}

\begin{itemize}
\item A first-class object possesses the properties such that it:
\begin{itemize}
\item May be passed as an argument
\item May be returned as a function value
\item May be stored in a data structure
\end{itemize}
\item Functions are first-class objects in functional languages
\item A higher-order function of a functional language is a function that:
\begin{itemize}
\item Accepts function arguments
\item Returns function values
\end{itemize}
\end{itemize}

\section{Functions as Arguments}

\begin{itemize}
\item Functions in Haskell may be passed as arguments to other functions:
\begin{lstlisting}
f :: Type1->Type2
g :: (Type1->Type2)->Type3
\end{lstlisting}
\item In the above, f could be passed as an argument to g
\item There are several higher-order functions implementing recursion patterns
\item Fundamental pattern of recursion: Accumulation
\begin{itemize}
\begin{lstlisting}
foldr :: (a->b->b)->b->[a]->b
\end{lstlisting}
\item Accepts a function as input and folds folds it between list elements
\item foldr1 does not need an identity element but fails on empty lists
\end{itemize}
\item Fundamental pattern of recursion: Transformation
\begin{itemize}
\begin{lstlisting}
map :: (a->b)->[a]->[b]
\end{lstlisting}
\item Accepts a function as input and applies it to every element of the list
\end{itemize}
\item Fundamental pattern of recursion: Selection
\begin{itemize}
\begin{lstlisting}
filter :: (a->Bool)->[a]->[a]
\end{lstlisting}
\item Constructs a new list from elements satisfying the input function
\end{itemize}
\end{itemize}

\section{Functions as Values}

\begin{itemize}
\item Composition of functions in Haskell is defined by (.) :
\begin{lstlisting}
f = g . h
\end{lstlisting}
\item The above would first apply any input to f to h, then g
\item (.) is therefore defined as:
\begin{lstlisting}
(.) :: (b->c)->(a->b)->(a->c)
\end{lstlisting}
\item The type of the second function must match the argument type of the first
\item The function id is defined as follows:
\begin{lstlisting}
id :: a->a
id a = a
\end{lstlisting}
\item Since f . id == f, id can be a base case for some recursive functions
\item Consider the following, which returns a function applied $n$ times:
\begin{lstlisting}
iter :: Int->(a->a)->(a->a)
iter n f = foldr (.) id (replicate n f)
\end{lstlisting}
\item The above unifies the concept of a function being a first-class object
\item We can also use the where construction to return functions as values:
\begin{lstlisting}
f :: Int->(Int->Int)
f x	= g
	where
	g y = x+y
\end{lstlisting}
\item Lambda notation is a simpler way of defining a function to return:
\begin{lstlisting}
f :: Int->(Int->Int)
f x = (\y -> x+y)
\end{lstlisting}
\item Lambda-defined functions are known as anonymous functions
\item Any function accepting two or more arguments may be partially applied:
\begin{lstlisting}
multiply :: Int->Int->Int
multiply x y = x*y

doubleAll :: [Int]->[Int]
doubleAll = map (multiply 2)
\end{lstlisting}
\item In actuality, every function in Haskell takes exactly one argument
\item The partially applied function is then applied to the other arguments
\end{itemize}

\section{Overloading and Type Classes}

\begin{itemize}
\item Polymorphic functions have the same behaviour for all types
\item Overloaded functions act differently for each type
\item The set of types over which a function is defined is called a type class
\item Members of a type class are called instances of that type class
\item A polymorphic function may be constrained to the type class Eq as follows:
\begin{lstlisting}
f :: Eq a => a -> a -> Bool
f x y = (x==y)
\end{lstlisting}
\item Since f is defined only for instances of Eq, (==) is always defined in f
\item The part before the $=>$ is called the context of the function
\item Multiple constraints on different types may be levelled in the context
\item Custom classes may be declared as follows:
\begin{lstlisting}
class Name a where
	f :: a -> b
	g :: a -> c
\end{lstlisting}
\item The declaration defines the name of the class and then the signature
\item A type is an instance of the class if the signature functions are defined:
\begin{lstlisting}
instance Name Int where
	f x = x
	g x = (h x)
\end{lstlisting}
\item Signature functions may be defined in terms of each other by default
\item Classes may also derive from other classes; consider the Ord class:
\begin{lstlisting}
class Eq a => Ord a where
	(<), (<=), (>), (>=)	:: a -> a -> Bool
	max, min		:: a -> a -> a
	compare			:: a -> a -> Ordering
\end{lstlisting}
\item The Ord class inherits all the operations of the Eq class
\end{itemize}

\section{Built-in Haskell Classes}

\begin{itemize}
\item Equality (Eq):
\begin{lstlisting}
class Eq a where
 (==), (/=) :: a -> a -> Bool
\end{lstlisting}
\item Ordering (Ord):
\begin{lstlisting}
class (Eq a) => Ord a where
	(<), (<=), (>), (>=)	:: a -> a -> Bool
	max, min		:: a -> a -> a
	compare			:: a -> a -> Ordering
\end{lstlisting}
\item Enumeration (Enum):
\begin{lstlisting}
class (Ord a) => Enum a where
	toEnum		:: Int -> a
	fromEnum	:: a -> Int
	enumFrom	:: a -> [a]
	enumFromThen	:: a -> a -> [a]
	enumFromTo	:: a -> a -> [a]
	enumFromThenTo	:: a -> a -> a -> a -> [a]
\end{lstlisting}
\item Bounded types (Bounded):
\begin{lstlisting}
class Bounded a where
	minBound, maxBound :: a
\end{lstlisting}
\item From values to strings (Show):
\begin{lstlisting}
class Show a where
	showsPrec	:: Int -> a -> ShowS
	show		:: a -> String
	showList	:: [a] -> ShowS
\end{lstlisting}
\item From strings to values (Read):
\begin{lstlisting}
class Read a where
	read	:: String -> a
\end{lstlisting}
\end{itemize}

\section{Algebraic Types}

\begin{itemize}
\item Algebraic types are generalized language constructs
\item The most basic algebraic type is the enumerated type:
\begin{lstlisting}
data Bool = True | False
\end{lstlisting}
\item Enumerated types have values limited to their explicit definition
\item True and False are the constructors of type Bool
\item Product types amalgamate other types:
\begin{lstlisting}
data People = Person Name Age
\end{lstlisting}
\item Person is a constructor for People, and may be represented thus:
\begin{lstlisting}
Person :: Name -> Age -> People
\end{lstlisting}
\item A product type may have multiple constructors
\item An algebraic type can be defined to be an instance of any class
\item The system may be asked to supply instantiations of classes as follows:
\begin{lstlisting}
data Season = Spring | Summer | Autumn | Winter
		deriving (Eq, Ord, Show, Read)
\end{lstlisting}
\item Defining algebraic types recursively allows for much greater expression
\item Consider the recursive algebraic type of a list of integers:
\begin{lstlisting}
data IntList = EmptyList | ConsPair Int IntList
\end{lstlisting}
\item In the above, EmptyList is [ ] and ConsPair is equivalent to (:)
\item The list $[1,2,3]$ would be represented thus:
\begin{lstlisting}
(ConsPair 1 (ConsPair 2 (ConsPair 3 EmptyList)))
\end{lstlisting}
\item Constructors may also be used in infix form:
\begin{lstlisting}
(1 'ConsPair' (2 'ConsPair' (3 'ConsPair' EmptyList)))
\end{lstlisting}
\item Constructors may also be defined and then used in infix form sans quotes:
\begin{lstlisting}
data IntList = EmptyList | Int :ConsPair: IntList
\end{lstlisting}
\item Two mutually recursive algebraic types use each other in their definition
\item Polymorphism may also be used when defining algebraic types:
\begin{lstlisting}
data List a = EmptyList | ConsPair a IntList
\end{lstlisting}
\end{itemize}

\section{Induction on Algebraic Types}

\begin{itemize}
\item For structural induction over an algebraic type:
\begin{itemize}
\item Construct a base case for each nonrecursive constructor
\item Construct an inductive hypothesis for each recursive constructor
\end{itemize}
\item If a constructor recurses on multiple values, assume their conjunction
\end{itemize}

\section{Lazy Evaluation}

\begin{itemize}
\item Expressions in Haskell are not evaluated until they are needed
\item If expressions are not needed, they will not be evaluated at all
\item A duplicated expression is never evaluated more than once
\item Expressions are evaluated from outside in and left to right
\item A construct taking advantage of this is infinite lists
\item An infinite list is defined but only evaluated to $n$ when requested
\end{itemize}

\end{document}
