\documentclass{article}
\usepackage{parskip}
\usepackage{amsmath}
\usepackage{amssymb}
\usepackage{listings}

\setlength{\parindent}{0cm}

\begin{document}

\lstset{language=Prolog, frame = single}

\title{CPSC 449 - Prolog}
\author{Andrew Helwer}
\date{Winter 2011}
\maketitle

\section{Prolog Basics}

\begin{itemize}
\item Short for PROgramming LOGic
\item Prolog revolves around facts and queries on those facts
\item Three fundamental concepts:
\begin{itemize}
\item Facts - things which are unconditionally true
\item Rules - things which are conditionally true
\item Queries - questions answered using facts and rules
\end{itemize}
\item Fact example:
\begin{lstlisting}
girl(allison).
father(don, allison).
\end{lstlisting}
\item Rule example:
\begin{lstlisting}
father(don, allison) :- daughter(allison, don).
\end{lstlisting}
\item If Don is Allison's father, then Allison is Don's daughter
\item Logical And is represented with , (comma) logical Or with ; (semicolon)
\begin{lstlisting}
daughter(allison, don) :- 
	father(don, allison), 
	girl(allison).
\end{lstlisting}
\item Variables are specified by starting with an upper-case letter
\begin{lstlisting}
?- girl(X).
\end{lstlisting}
\item This will return allison in the current knowledge base
\item ; (semicolon) will cause Prolog to backtrack and re-satisfy a variable
\item The ; is a way of getting Prolog to simulate nondeterminism
\end{itemize}

\section{Pattern Matching and Recursion}

\begin{itemize}
\item Patterns in Prolog are checked top to bottom, left to right
\item Recursive rules are an important part of Prolog:
\begin{lstlisting}
ancestor(X,Y) :- parent(X,Y).
ancestor(X,Y) :- parent(Z,Y), ancestor(X,Z).
\end{lstlisting}
\item parent is the base case, while ancestor is the general case
\item Lists are also an important part of Prolog
\item They are represented by a head element followed by the tail
\item $[\:]$ represents the empty list
\begin{lstlisting}
elemOf(X, [X | T]).
elemOf(X, [H | T]) :- member(X, T).
\end{lstlisting}
\item If Prolog cannot satisfy a query, it will return false
\item Only the cases where a relationship is true need be considered
\end{itemize}


\end{document}
